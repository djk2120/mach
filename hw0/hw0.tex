\documentclass[11pt, oneside]{article}   	% use "amsart" instead of "article" for AMSLaTeX format
\usepackage{geometry}                		% See geometry.pdf to learn the layout options. There are lots.
\geometry{letterpaper}                   		% ... or a4paper or a5paper or ... 
%\geometry{landscape}                		% Activate for for rotated page geometry
\usepackage[parfill]{parskip}    		% Activate to begin paragraphs with an empty line rather than an indent
\usepackage{graphicx}				% Use pdf, png, jpg, or eps§ with pdflatex; use eps in DVI mode
								% TeX will automatically convert eps --> pdf in pdflatex		
\usepackage{amssymb}
\usepackage{amsmath}
\title{Xylem Water Balance}
\author{Daniel Kennedy}
%\date{}							% Activate to display a given date or no date

\begin{document}
\maketitle
%\section{}
%\subsection{}

In this update of CLM hydraulic controls, plant hydraulics are resolved.

The model features two canopy layers, for shaded and unshaded leaves

Transpiration from these layers are:

\begin{equation}
T_{sunlit}=lai_{sunlit}*vpd_{sunlit}*g_{s,max,sunlit}*f_{sto}\left(\psi_{sto,1}\right)
\end{equation}
\begin{equation}
T_{shaded}=lai_{shaded}*vpd_{shaded}*g_{s,max,shaded}*f_{sto}\left(\psi_{sto,2}\right)
\end{equation}

The fluxes through the plant are
\begin{equation}
q_{branch,1}=lai_{sunlit}*k_{l,max}*f_l\left(\psi_{x}\right)*\left(\psi_x-\psi_{sto,1}\right)
\end{equation}
\begin{equation}
q_{branch,2}=lai_{shaded}*k_{l,max}*f_l\left(\psi_{x}\right)*\left(\psi_x-\psi_{sto,2}\right)
\end{equation}
\begin{equation}
q_{stem}=sai*k_{x,max}/z*f_x\left(\psi_{r}\right)*\left(\psi_r-\psi_{x}-\rho gz\right)
\end{equation}
\begin{equation}
q_{soil,1}=rai*k_{r,max}*f_r\left(\psi_{soil,1}\right)*\left(\psi_{soil,1}-\psi_{r}\right)
\end{equation}
\begin{equation}
q_{soil,2}=rai*k_{r,max}*f_r\left(\psi_{soil,2}\right)*\left(\psi_{soil,2}-\psi_{r}\right)
\end{equation}
...
\begin{equation}
q_{soil,i}=rai*k_{r,max}*f_r\left(\psi_{soil,i}\right)*\left(\psi_{soil,i}-\psi_{r}\right)
\end{equation}

The hydraulics are resolved each timestep in the explicit way, where
\begin{equation}
q^{t+1}=q^{t}+\dfrac{\delta q}{\delta \psi} \Delta \psi
\end{equation}

Using the equation set:
\begin{equation}
T_{sunlit}^{t+1}=q_{branch,1}^{t+1}
\end{equation}
\begin{equation}
T_{shaded}^{t+1}=q_{branch,2}^{t+1}
\end{equation}
\begin{equation}
q_{branch,1}^{t+1}+q_{branch,2}^{t+1}=q_{stem}^{t+1}
\end{equation}
\begin{equation}
q_{stem}^{t+1}=\sum\limits_{i=1}^{nlevsoi}q_{soil,i}^{t+1}
\end{equation}
\begin{equation}
q_i-q_{i-1}-e_i=\dfrac{\delta S_i}{\delta t}
\end{equation}
\begin{equation}
A=
\left[ \begin {array}{cccc} 
\dfrac{\delta q_{b,1}}{\delta \psi_{sto,1}}-\dfrac{\delta T_{sun}}{\delta \psi_{sto,1}}
&&\dfrac{\delta q_{b,1}}{\delta \psi_{x}}
\cr &\dfrac{\delta q_{b,2}}{\delta \psi_{sto,2}}-\dfrac{\delta T_{sha}}{\delta \psi_{sto,2}}
&\dfrac{\delta q_{b,2}}{\delta \psi_{x}}
\cr -\dfrac{\delta q_{b,1}}{\delta \psi_{sto,1}}
&-\dfrac{\delta q_{b,2}}{\delta \psi_{sto,1}}
&\dfrac{\delta q_{stem}}{\delta \psi_{x}}-\dfrac{\delta q_{b,1}}{\delta \psi_{x}}-\dfrac{\delta q_{b,1}}{\delta \psi_{x}}
&\dfrac{\delta q_{stem}}{\delta \psi_{r}}
\cr&&-\dfrac{\delta q_{stem}}{\delta \psi_{x}}
&\dfrac{\delta q_{soil}}{\delta \psi_{r}}-\dfrac{\delta q_{stem}}{\delta \psi_{r}}
\end {array} \right]
\end{equation}


\break
For the fourth balancing equation


\begin{equation}
q_{stem}^{t+1}=\sum\limits_{i=1}^{nlevsoi}q_{soil,i}^{t+1}
\end{equation}
\begin{equation}
q_{stem}^t-q_{soil}^t=\left(\dfrac{\delta q_{soil}}{\delta \psi}-\dfrac{\delta q_{stem}}{\delta \psi}\right)\Delta \psi
\end{equation}
where
\begin{equation}
q_{stem}=sai*k_{x,max}/z*f_x\left(\psi_{r}\right)*\left(\psi_r-\psi_{x}-\rho gz\right)
\end{equation}
\begin{equation}
q_{soil,i}=rai*k_{r,max}*f_r\left(\psi_{soil,i}\right)*\left(\psi_{soil,i}-\psi_{r}\right)
\end{equation}

yielding linear terms

\begin{equation}
\dfrac{\delta q_{stem}}{\delta \psi_x}=-sai*k_{x,max}/z*f_x\left(\psi_{r}\right)
\end{equation}

\begin{equation}
\dfrac{\delta q_{stem}}{\delta \psi_r}=sai*k_{x,max}/z*\dfrac{\delta f_x}{\delta\psi_r}\left(\psi_{r}\right)*\left(\psi_r-\psi_{x}-\rho gz\right)+sai*k_{x,max}/z*f_x\left(\psi_{r}\right)
\end{equation}

\begin{equation}
\dfrac{\delta q_{soil}}{\delta \psi_r}=-rai*k_{r,max}*f_r\left(\psi_{soil,i}\right)
\end{equation}

and quadratic terms

\begin{equation}
\dfrac{\delta^2 q_{stem}}{\delta \psi_x\delta\psi_r}=-sai*k_{x,max}/z*\dfrac{\delta f_x}{\delta\psi_r}\left(\psi_{r}\right)
\end{equation}

\begin{equation}
\dfrac{\delta^2 q_{stem}}{\left(\delta \psi_r\right)^2}=sai*k_{x,max}/z*\dfrac{\delta^2 f_x}{\left(\delta\psi_r\right)^2}\left(\psi_{r}\right)*\left(\psi_r-\psi_{x}-\rho gz\right)+2*sai*k_{x,max}/z*\dfrac{\delta f_x}{\delta\psi_r}\left(\psi_{r}\right)
\end{equation}

\begin{equation}
\end{equation}







$q_i$ is the flux of water to the layer below

$q_{i-1}$ is the flux of water from the layer above

$e_i$ is the transpiration sink term

$S$ is the plant water storage

and $t$ represents time

As this is the top layer, there is no flux of water from above ($q_{i-1}=0$). The flux from below, $q_i$, is equal to the basal flow, $U$. The transpiration sink term is equal to the total transpirational loss, $T$. Thus the balance at time step $N+1$ is:
\begin{equation} \label{eq:a}
U^{N+1}-T^{N+1}=\dfrac{\delta S}{\delta t}
\end{equation}
The change in plant water storage, $\Delta S$, is equal to the product of vegetation height and the change in volumetric water content: $\Delta z_{veg} \Delta \theta_{veg}$
\begin{equation}
U^{N+1}-T^{N+1}=\Delta z_{veg} \dfrac{\Delta \theta_{veg}}{\Delta t}
\end{equation}
The basal flow can be linearized about $\theta$ using a Taylor series expansion as
\begin{equation}
U^{N+1} = U^{N} + \dfrac{\delta U}{\delta \theta}\Delta\theta
\end{equation}
Which can be substituted into equation~\ref{eq:a} to yield after rearranging:
\begin{equation}
U^{N}-T^{N+1}=\Delta z_{veg} \dfrac{\Delta \theta_{veg}}{\Delta t}-\dfrac{\delta U}{\delta \theta}\Delta\theta
\end{equation}
$U$ is the summation of transpiration sink terms from all the soil layers, whereby it depends on \{$\theta_{veg}$, $\theta_1$, $\theta_2$, ...\}, such that
\begin{equation}
U^{N}-T^{N+1}=\Delta z_{veg} \dfrac{\Delta \theta_{veg}}{\Delta t}-\dfrac{\delta U}{\delta \theta_{veg}}\Delta\theta_{veg}-\dfrac{\delta U}{\delta \theta_1}\Delta\theta_1-\dfrac{\delta U}{\delta \theta_2}\Delta\theta_2 \:...
\end{equation}
Recalling that $U=\sum\limits_{i=1}^n e_i$ and 
\begin{equation}
e_i=-r_{i}k\left[\psi_i\right]\left[\dfrac{\left( \psi_x-\psi_i \right) +\left( \psi_{E,i}-\psi_{E,x} \right)}{z_x-z_i}\right]
\end{equation}
The dependence of $U$ on $\theta$ in any one of the soil layers i=\{1, 2, ... , $nlevsoi+1$\} is only due to changes in that layer's transpiration sink term $e_i$. As such, the derivative of the basal flow rate with respect to the change in volumetric soil water in soil layer $i$ will be:
\begin{equation}
\dfrac{\delta U}{\delta \theta_i}=\dfrac{\delta e_i}{\delta\theta_{i}}=r_{i}\left[\dfrac{k\left[\psi_i\right]}{z_i-z_x}\dfrac{\delta \psi_i}{\delta \theta_{liq,i}}\right]-r_{i}\dfrac{\delta k \left[ \psi_i \right]}{\delta \theta_{liq,i}} \left[ \dfrac{\left( \psi_x - \psi_i \right) + \left( \psi_{E,i}-\psi_{E,x}\right)}{z_i-z_x}\right]
\end{equation}
Where
\begin{equation}
\dfrac{\delta \psi_i}{\delta \theta_{liq,i}} = -B_i \dfrac{\psi_i}{\theta_i}
\end{equation}
\begin{equation}
k\left[\psi_i\right] = k_{max}exp\left(-\left(\dfrac{\psi_i}{\psi_{50}}\right)^{c_k}\right)
\end{equation}
\begin{equation}
\dfrac{\delta k \left[ \psi_i \right]}{\delta \theta_{liq,i}} =
\end{equation}
The derivative of $U$ with respect to $\theta_{veg}$ is a summation, because every layer's transpiration sink term depends on $\theta_{veg}$.
\begin{equation}
\dfrac{\delta U}{\delta \theta_{veg}}=\sum\limits_{i=1}^{nlevsoi}\dfrac{\delta e_i}{\delta\theta_{veg}}=\sum\limits_{i=1}^{nlevsoi}-r_{i}\left[\dfrac{k\left[\psi_i\right]}{z_i-z_x}\dfrac{\delta \psi_x}{\delta \theta_{veg}}\right]
\end{equation}
where $\dfrac{\delta \psi_{x}}{\delta \theta_{veg}} = C\left[\psi_x\right]$

\end{document}


